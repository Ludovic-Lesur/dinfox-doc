\section{Global architecture}

\subsection{Introduction}

The main goal of the DINFox project is to design a scalable power system in order to control and monitor any kind of power supply (AC line, solar panel, backup, regulators, relays, etc.).
\medskip \\
This main idea can be split in several objectives:

\bulletlist{
    \item Design a \textbf{single module for a given function} (1 board for 1 relay for instance) and use a \textbf{common physical bus} to perform the communication between boards. Thanks to this modular approach, each DINFox system can be composed of a custom group of nodes according to the required features.
    \item Provide a mechanical design in such a way that the boards can be fixed to a DIN rail and can be stacked together without additional wiring.
    \item Define a \textbf{unified interface} to access all nodes in order to manage a  \textbf{single API for local control and monitoring}.
    \item Achieve the \textbf{minimum power consumption} to be able to monitor backup sources such as battery powered systems.
}

\subsection{Functional description}

A DINFox system consists of several DIN rail modules, called \textbf{nodes}, physically connected through an RS485 bus. The general behavior of the system is controlled by one \textbf{master} module which is the master of the RS485 bus. All other nodes are \textbf{slaves}, they are continuously listening on the bus to wait for commands.
\medskip \\
The DINFox system is a concrete implementation of the \textbf{Unified Node Access (UNA)} framework: all nodes of the bus are controlled through a set of registers regardless of the physical interface.
\bigskip

\begin{figure}[h]
    \centering
    \graph{\linewidth}{dinfox-system.drawio.png}
    \vspace*{\baselineskip}
    \capt{Functional architecture of a DINFox system}
\end{figure}

\newpage